\chapter{Related Work}

\section{Structured Concurrency}
Structured concurrency is a programming paradigm that aims to simplify reasoning about concurrent programs. This is done by encapsulating threads in control flow constructs that ensure termination. \todo{this hsould probably be in background instead}


I.e. Determinism, performance, extension of existing languages, Expressiveness, Annotation overhead, deadlock freedom

\section{Isolation}
A prerequisite for safe concurrency is isolation of memory areas. In LaCasa \todo{reference} by Haller a region-based approach is pursued. By encapusa


Ownership based, region based, annotations etc.


In DPJ \todo{reference} a region-based type and effect system is developed to guarantee deterministic semantics through static checking. It is backwards compatible with Java and supports gradually adding concurrency.

To ensure this, the heap is partitioned into regions such that no overlaps between regions exist. Variables are annotated with the region they reside in.


Reference Capabilities for Flexibe memory management
Deterministic concurrency using lattices

Consistency types for replicated data
Philipps PhD students work

LaCasa

Integrating Task Parallelism with Actors


Odersky, probably outdated:
Scoped Capabilities
Safer Exceptions

Calculus of atomic actions
Relevant for proving determinism, we did it simpler.
