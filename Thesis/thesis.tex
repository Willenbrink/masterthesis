
%%
%% forked from https://gits-15.sys.kth.se/giampi/kthlatex kthlatex-0.2rc4 on 2020-02-13
%% expanded upon by Gerald Q. Maguire Jr.
%% This template has been adapted by Anders Sjögren to the University
%% Engineering Program in Computer Science at KTH ICT. Adaptation is the
%% translation of English headings into Swedish as the addition of Swedish


%% The template is designed to handle a thesis in English or Swedish
% set the default language to english or swedish by passing an option to the documentclass - this handles the inside tile page
% To optimize for digital output (this changes the color palette add the option: digitaloutput
% To use \ifnomenclature add the option nomenclature
% To use bibtex or biblatex - include one of these as an option
\documentclass[nomenclature, english, bibtex]{kththesis}
%\documentclass[swedish, biblatex]{kththesis}
% if pdflatex \usepackage[utf8]{inputenc}

%% Conventions for todo notes:
% Informational
%% \generalExpl{Comments/directions/... in English}
\newcommand*{\generalExpl}[1]{\todo[inline]{#1}}                

% Language specific information (currently in English or Swedish)
\newcommand*{\engExpl}[1]{\todo[inline, backgroundcolor=kth-lightgreen40]{#1}} %% \engExpl{English descriptions about formatting}
\newcommand*{\sweExpl}[1]{\todo[inline, backgroundcolor=kth-lightblue40]{#1}}  %% % \sweExpl{Text på svenska}

% warnings
\newcommand*{\warningExpl}[1]{\todo[inline, backgroundcolor=kth-lightred40]{#1}} %% \warningExpl{warnings}

% Uncomment to hide specific comments, to hide **all** ToDos add `final` to
% document class
% \renewcommand\warningExpl[1]{}
% \renewcommand\generalExpl[1]{}
% \renewcommand\engExpl[1]{}
% For example uncommenting the following line hides the Swedish language explanations
% \renewcommand\sweExpl[1]{}


% \usepackage[style=numeric,sorting=none,backend=biber]{biblatex}
\ifbiblatex
    %\usepackage[language=english,bibstyle=authoryear,citestyle=authoryear, maxbibnames=99]{biblatex}
    % alternatively you might use another style, such as IEEE and use citestyle=numeric-comp  to put multiple citations in a single pair of square brackets
    %\usepackage[style=ieee,citestyle=numeric-comp]{biblatex}
    \addbibresource{references.bib}
    %\DeclareLanguageMapping{norsk}{norwegian}
\else
    % The line(s) below are for BibTeX
    \bibliographystyle{bibstyle/myIEEEtran}
    %\bibliographystyle{apalike}
\fi


% include a variety of packages that are useful
\input{lib/includes}
\input{lib/kthcolors}

%\glsdisablehyper
%\makeglossaries
%\makenoidxglossaries
%\input{lib/acronyms}                %load the acronyms file

\input{lib/defines}  % load some additional definitions to make writing more consistent

% The following is needed in conjunction with generating the DiVA data with abstracts and keywords using the scontents package and a modified listings environment
%\usepackage{listings}   %  already included
\ExplSyntaxOn
\newcommand\typestoredx[2]{\expandafter\__scontents_typestored_internal:nn\expandafter{#1} {#2}}
\ExplSyntaxOff
\makeatletter
\let\verbatimsc\@undefined
\let\endverbatimsc\@undefined
\lst@AddToHook{Init}{\hyphenpenalty=50\relax}
\makeatother


\lstnewenvironment{verbatimsc}
    {
    \lstset{%
        basicstyle=\ttfamily\tiny,
        backgroundcolor=\color{white},
        %basicstyle=\tiny,
        %columns=fullflexible,
        columns=[l]fixed,
        language=[LaTeX]TeX,
        %numbers=left,
        %numberstyle=\tiny\color{gray},
        keywordstyle=\color{red},
        breaklines=true,                 % sets automatic line breaking
        breakatwhitespace=true,          % sets if automatic breaks should only happen at whitespace
        %keepspaces=false,
        breakindent=0em,
        %fancyvrb=true,
        frame=none,                     % turn off any box
        postbreak={}                    % turn off any hook arrow for continuation lines
    }
}{}

%% Add some more keyowrds to bring out the structure more
\lstdefinestyle{[LaTeX]TeX}{
morekeywords={begin, todo, textbf, textit, texttt}
}

%% definition of new command for bytefield package
\newcommand{\colorbitbox}[3]{%
	\rlap{\bitbox{#2}{\color{#1}\rule{\width}{\height}}}%
	\bitbox{#2}{#3}}




% define a left aligned table cell that is ragged right
\newcolumntype{L}[1]{>{\raggedright\let\newline\\\arraybackslash\hspace{0pt}}p{#1}}

% Because backref is not compatible with biblatex
\ifbiblatex
    \usepackage[plainpages=false]{hyperref}
\else
    \usepackage[
    backref=page,
    pagebackref=false,
    plainpages=false,
                            % PDF related options
    unicode=true,           % Unicode encoded PDF strings
    bookmarks=true,         % generate bookmarks in PDF files
    bookmarksopen=false,    % Do not automatically open the bookmarks in the PDF reading program
    pdfpagemode=UseNone,    % None, UseOutlines, UseThumbs, or FullScreen
    destlabel,              % better naming of destinations
    ]{hyperref}
    \usepackage{backref}
    %
    % Customize list of backreferences.
    % From https://tex.stackexchange.com/a/183735/1340
    \renewcommand*{\backref}[1]{}
    \renewcommand*{\backrefalt}[4]{%
    \ifcase #1%
          \or [Page~#2.]%
          \else [Pages~#2.]%
    \fi%
    }
\fi
\usepackage[all]{hypcap}	%% prevents an issue related to hyperref and caption linking

%% Acronyms
% note that nonumberlist - removes the cross references to the pages where the acronym appears
% note that super will set the descriptions text aligned
% note that nomain - does not produce a main glossary, thus only acronyms will be in the glossary
% note that nopostdot - will prevent there being a period at the end of each entry
\usepackage[acronym, style=super, section=section, nonumberlist, nomain,
nopostdot]{glossaries}
\setlength{\glsdescwidth}{0.75\textwidth}
\usepackage[]{glossaries-extra}
\ifinswedish
    %\usepackage{glossaries-swedish}
\fi

%% For use with the README_notes
% Define a new type of glossary so that the acronyms defined in the README_notes document can be distinct from those in the thesis template
% the tlg, tld, and dn will be the file extensions used for this glossary
\newglossary[tlg]{readme}{tld}{tdn}{README acronyms}


\input{lib/includes-after-hyperref}

%\glsdisablehyper
\makeglossaries
%\makenoidxglossaries
\input{lib/acronyms}                %load the acronyms file

% insert the configuration information with author(s), examiner, supervisor(s), ...
\input{custom_configuration}

\title{This is the title in the language of the thesis}
\subtitle{A subtitle in the language of the thesis}

% give the alternative title - i.e., if the thesis is in English, then give a Swedish title
\alttitle{Detta är den svenska översättningen av titeln}
\altsubtitle{Detta är den svenska översättningen av undertiteln}
% alternative, if the thesis is in Swedish, then give an English title
%\alttitle{This is the English translation of the title}
%\altsubtitle{This is the English translation of the subtitle}

% Enter the English and Swedish keywords here for use in the PDF meta data _and_ for later use
% following the respective abstract.
% Try to put the words in the same order in both languages to facilitate matching. For example:
\EnglishKeywords{Canvas Learning Management System, Docker containers, Performance tuning}
\SwedishKeywords{Canvas Lärplattform, Dockerbehållare, Prestandajustering}

%%%%% For the oral presentation
%% Add this information once your examiner has scheduled your oral presentation
\presentationDateAndTimeISO{2022-03-15 13:00}
\presentationLanguage{eng}
\presentationRoom{via Zoom https://kth-se.zoom.us/j/ddddddddddd}
\presentationAddress{Isafjordsgatan 22 (Kistagången 16)}
\presentationCity{Stockholm}

% When there are multiple opponents, separate their names with '\&'
% Opponent's information
\opponentsNames{A. B. Normal \& A. X. E. Normalè}

% Once a thesis is approved by the examiner, add the TRITA number
% The TRITA number for a thesis consists of two parts a series (unique to each school)
% and the number in the series which is formatted as the year followed by a colon and
% then a unique series number for the thesis - starting with 1 each year.
\trita{TRITA-EECS-EX}{2022:00}

% Put the title, author, and keyword information into the PDF meta information
\input{lib/pdf_related_includes}


% the custom colors and the commands are defined in defines.tex    
\hypersetup{
	colorlinks  = true,
	breaklinks  = true,
	linkcolor   = \linkscolor,
	urlcolor    = \urlscolor,
	citecolor   = \refscolor,
	anchorcolor = black
}

\ifnomenclature
% The following lines make the page numbers and equations hyperlinks in the Nomenclature list
\renewcommand*{\pagedeclaration}[1]{\unskip, \dotfill\hyperlink{page.#1}{page\nobreakspace#1}}
% The following does not work correctly, as the name of the cross-reference is incorrect
%\renewcommand*{\eqdeclaration}[1]{, see equation\nobreakspace(\hyperlink{equation.#1}{#1})}

% You can also change the page heading for the nomenclature
\renewcommand{\nomname}{List of Symbols Used}

% You can even add customization text before the list
\renewcommand{\nompreamble}{The following symbols will be later used within the body of the thesis.}
\makenomenclature
\fi

%
% The commands below are to configure JSON listings
% 
% format for JSON listings
\colorlet{punct}{red!60!black}
\definecolor{delim}{RGB}{20,105,176}
\definecolor{numb}{RGB}{106, 109, 32}
\definecolor{string}{RGB}{0, 0, 0}

\lstdefinelanguage{json}{
    numbers=none,
    numberstyle=\small,
    frame=none,
    rulecolor=\color{black},
    showspaces=false,
    showtabs=false,
    breaklines=true,
    postbreak=\raisebox{0ex}[0ex][0ex]{\ensuremath{\color{gray}\hookrightarrow\space}},
    breakatwhitespace=true,
    basicstyle=\ttfamily\small,
    extendedchars=false,
    upquote=true,
    morestring=[b]",
    stringstyle=\color{string},
    literate=
     *{0}{{{\color{numb}0}}}{1}
      {1}{{{\color{numb}1}}}{1}
      {2}{{{\color{numb}2}}}{1}
      {3}{{{\color{numb}3}}}{1}
      {4}{{{\color{numb}4}}}{1}
      {5}{{{\color{numb}5}}}{1}
      {6}{{{\color{numb}6}}}{1}
      {7}{{{\color{numb}7}}}{1}
      {8}{{{\color{numb}8}}}{1}
      {9}{{{\color{numb}9}}}{1}
      {:}{{{\color{punct}{:}}}}{1}
      {,}{{{\color{punct}{,}}}}{1}
      {\{}{{{\color{delim}{\{}}}}{1}
      {\}}{{{\color{delim}{\}}}}}{1}
      {[}{{{\color{delim}{[}}}}{1}
      {]}{{{\color{delim}{]}}}}{1}
      {’}{{\char13}}1,
}

\lstdefinelanguage{XML}
{
  basicstyle=\ttfamily\color{blue}\bfseries\small,
  morestring=[b]",
  morestring=[s]{>}{<},
  morecomment=[s]{<?}{?>},
  stringstyle=\color{black},
  identifierstyle=\color{blue},
  keywordstyle=\color{cyan},
  breaklines=true,
  postbreak=\raisebox{0ex}[0ex][0ex]{\ensuremath{\color{gray}\hookrightarrow\space}},
  breakatwhitespace=true,
  morekeywords={xmlns,version,type}% list your attributes here
}

% In case you use both listings and lstlistings - this makes them both use the same counter
\makeatletter
\AtBeginDocument{\let\c@listing\c@lstlisting}
\makeatother
\usepackage{subfiles}

% To have Creative Commons (CC) license and logos use the doclicense package
% Note that the lowercase version of the license has to be used in the modifier
% i.e., one of by, by-nc, by-nd, by-nc-nd, by-sa, by-nc-sa, zero.
% For background see:
% https://www.kb.se/samverkan-och-utveckling/oppen-tillgang-och-bibsamkonsortiet/open-access-and-bibsam-consortium/open-access/creative-commons-faq-for-researchers.html
% https://kib.ki.se/en/publish-analyse/publish-your-article-open-access/open-licence-your-publication-cc
\begin{comment}
\usepackage[
    type={CC},
    %modifier={by-nc-nd},
    %version={4.0},
    modifier={by-nc},
    imagemodifier={-eu-88x31},  % to get Euro symbol rather than Dollar sign
    hyphenation={RaggedRight},
    version={4.0},
    %modifier={zero},
    %version={1.0},
]{doclicense}
\end{comment}



%% TODO Added by me
\usepackage{semantic}
\usepackage{amssymb}
\usepackage{amsthm}
\usepackage{xparse}

\theoremstyle{definition}
\newtheorem{definition}{Definition}


\begin{document}
%\selectlanguage{swedish}
%
\selectlanguage{english}

%%% Set the numbering for the title page to a numbering series not in the preface or body
\pagenumbering{alph}
\kthcover
\clearpage\thispagestyle{empty}\mbox{} % empty back of front cover
\titlepage

% If you do not want to have a bookinfo page, comment out the line saying \bookinfopage and add a \cleardoublepage
% If you want a bookinfo page: you will get a copyright notice, unless you have used the doclicense package in which case you will get a Creative Commons license. To include the doclicense package, uncomment the configuration of this package above and configure it with your choice of license.
\bookinfopage

% Frontmatter includes the abstracts and table-of-contents
\frontmatter
\setcounter{page}{1}
\section{Inference Rules}

% We have two kind of frames: The normal stack frames (sframe) and finish frames (fframe). They take a number of arguments:
% s:aligned? OL:variable assignments m:term OP:Permissions Ol:label
\NewDocumentCommand{\sframe}{ s O{L} m O{P} O{l} }{\langle #2,\ \IfBooleanT{#1}{&&} #3,\ \IfBooleanT{#1}{&&} #4 \IfBooleanT{#1}{&&}\rangle^#5 \IfBooleanT{#1}{&&} }
%s:aligned? m:id
\NewDocumentCommand{\fframe}{ s m }{\IfBooleanT{#1}{&&} \langle \mathit{FINISH} #2 \rangle}

% We have three kinds of reductions: frame, stack and tasks. Each come with sensible defaults
\NewDocumentCommand{\reducframe}{m m}{%
  \begin{alignedat}{7}
    H, &#1\\\rightarrow\ H, &#2
  \end{alignedat}%
}
\NewDocumentCommand{\reducstack}{m O{FS} m O{FS}}{%
  \begin{alignedat}{7}
    H, &#1 \circ #2\\\twoheadrightarrow\ H, &#3 \circ #4
  \end{alignedat}%
}
\NewDocumentCommand{\reductasks}{m O{TS} m O{TS}}{%
  \begin{alignedat}{7}
    H, &\{#1\} && \uplus #2\\\leadsto\ H, &\{#3\} && \uplus #4
  \end{alignedat}%
}
\NewDocumentCommand{\reducasync}{m m}{%
  \begin{alignedat}{7}
    H, & #1 \\\leadsto\ H, & #2
  \end{alignedat}%
}

\subsection{Extension}
\subsubsection{Typing}
\inference[T-TASK]{x:C; \mathit{ocap} \vdash t : \tau & \Gamma; a \vdash b : Q \vartriangleright \mathit{Box}[C]}{\Gamma; a \vdash \mathit{task}(b) \{x \Rightarrow t\} : Q \vartriangleright \mathit{Task}[C]}
\inference[]{}{}

\inference[T-ASYNC]{\mathit{Perm}[Q] \in \Gamma & \Gamma \setminus \mathit{Perm}[Q]; a \vdash s : \sigma & \Gamma; a \vdash x : Q \vartriangleright \mathit{Task}[C]}{\Gamma; a \vdash \mathit{async}(x) \{s\} : \bot}
\inference[]{}{}

\inference[T-FINISH]{\Gamma; a \vdash t : \tau}{\Gamma; a \vdash \mathit{finish} \{ t \} : null}
\inference[]{}{}


\subsubsection{Evaluation}
\inference[E-TASK]{L(b') = b(o, p)}{%
  \reductasks{%
      (f, \sframe{%
        let\ x = \mathit{task}(b')\{x \Rightarrow t\}\ in\ s%
      })%
  }{%
      (f, \sframe[L[x \rightarrow \mathit{task}(b(o,p),t)]]{%
        s
      })%
  }%
}
\inference[]{}{}

\inference[E-ASYNC]{%
  L(y) = \mathit{task}(b(o,p),t) &
  p \in P \\
  T_1 = (f, \sframe{s}[P][\epsilon]) &
  T_2 = (f, \sframe[[x \rightarrow o]]{t}[\emptyset][\epsilon])
}{%
  \reducasync{
    \{(f,%
      \sframe{\mathit{async}(y)\{s\}} \circ FS
    )\} && \uplus TS%
    }{
    \{ T_1, T_2 \} && \uplus TS%
    }
}
\inference[]{}{}

\inference[E-FINISH1]{%
  T = (f', \sframe{t}[P][\epsilon]) &
  f' \mathit{fresh}
}{%
  \reductasks{%
    (f,%
      \sframe{ \mathit{let}\ x = \mathit{finish} \{\ t\ \}\ \mathit{in}\ s} \circ FS%
    )
  }{%
    (f,%
      \fframe{f'} \circ \sframe[L[x\rightarrow \mathit{null}]]{s} \circ FS%
    )
  }[\{T\} \uplus TS]
}
\inference[]{}{}

\inference[E-FINISH2]{%
  \nexists (f', FS) \in TS
}{%
  \reductasks{%
    (f, \fframe{f'} \circ FS )%
  }{%
    (f, FS )%
  }
}
\inference[]{}{}

\inference[E-TASK-DONE]{}{
  H,\epsilon \uplus TS \leadsto TS
}


\subsection{LaCasa}
\subsubsection{Well-Formedness}

\inference[WF-VAR]{%
  L(x) = \mathit{null} \vee \\%
  L(x) = o \wedge \mathit{typeof}(H,o) <: \Gamma(x) \vee \\%
  L(x) = b(o,p) \wedge \Gamma(x) = Q \vartriangleright Box[C] \wedge \mathit{typeof}(H,o) <: C%
}{H \vdash \Gamma ; L; x}

\inference[]{}{}
\inference[WF-PERM]{
  \gamma : permTypes(\Gamma) \longrightarrow P injective \\
  \forall x \in dom(\Gamma). \\
  (\Gamma(x) = Q \vartriangleright Box[C] \wedge L(x) = b(o,p) \wedge Perm[Q] \in \Gamma \vee \\
  \Gamma(x) = Q \vartriangleright Task[C] \wedge L(x) = task(b(o,p),t) \wedge Perm[Q] \in \Gamma) \\
  \Longrightarrow \gamma(Q) = p}{\vdash \Gamma; L; P}
\inference[]{}{}
\inference[WF-ENV]{dom(\Gamma) \subseteq dom(L) \\
\forall x \in dom(\Gamma). H \vdash \Gamma; L; x}{H \vdash \Gamma; L}
\inference[]{}{}
\inference[WF-METHOD1]{\Gamma_0, this : C, x : D; \epsilon \vdash t : E' & E' <: E}{C \vdash def m(x : D) : E = t}
\inference[]{}{}
\inference[WF-METHOD2]{\Gamma = \Gamma_0, this : C, x : Q \vartriangleright Box[D], Perm[Q] & Q fresh & \Gamma; \epsilon \vdash t : E' & E' <: E}{C \vdash def m(x : Box[D]) : E = t}
\inference[]{}{}
\inference[WF-PROGRAM]{p \vdash \bar{cd} & p \vdash \Gamma_0 & \Gamma_0; \epsilon \vdash t : \sigma}{p \vdash \bar{cd} \bar{vd} t}
\inference[]{}{}
\inference[WF-CLASS]{C \vdash \bar{md} & D = AnyRef \vee p \vdash class D ... \\
  \forall (def m ...) \in \bar{md}. override(m, C, D) \\
\forall var f : \sigma \in \bar{fd}. f \notin fields(D)}{p \vdash class C extends D \{\bar{fd} \bar{md}\}}
\inference[]{}{}
\inference[WF-OVERRIDE]{mtype(m, D) not defined \vee mtype(m, D) = mtype(m, C)}{override(m, C, D)}


\subsubsection{Typing}
\inference[T-NULL]{}{\Gamma; a \vdash null : Null}
\inference[]{}{}
\inference[T-VAR]{x \in dom(\Gamma)}{\Gamma; a \vdash x : \Gamma(x)}
\inference[]{}{}
\inference[T-LET]{\Gamma; a \vdash e : \tau & \Gamma, x : \tau; a \vdash t : \sigma}{\Gamma; a \vdash let x = e in t : \sigma}
\inference[]{}{}
\inference[T-SELECT]{\Gamma; a \vdash x : C & ftype(C, f) = D}{\Gamma; a \vdash x.f : D}
\inference[]{}{}
\inference[T-ASSIGN]{\Gamma; a \vdash x : C & ftype(C, f) = D \\ \Gamma; a \vdash y : D' & D' <: D}{\Gamma; a \vdash x.f = y : D}
\inference[]{}{}
\inference[T-INVOKE]{\Gamma; a \vdash x : C & mtype(C,m) = \sigma \rightarrow \tau \\
  \Gamma; a \vdash y : \sigma' & \sigma' <: \sigma \vee \\
(\sigma = Box[D] \wedge \sigma' = Q \vartriangleright Box[D] \wedge Perm[Q] \in \Gamma)}{\Gamma; a \vdash x.m(y) : \tau}
\inference[]{}{}
\inference[T-NEW]{a = ocap \Longrightarrow ocap(C) & \forall var f : \sigma \in \bar{fd} . \exists D. \sigma = D}{\Gamma; a \vdash new C : C}
\inference[]{}{}
\inference[T-OPEN]{\Gamma; a \vdash x : Q \vartriangleright Box[C] & Perm[Q] \in \Gamma & y : C; ocap \vdash t : \sigma}{\Gamma; a \vdash x.open \{y \Rightarrow t\} : Q \vartriangleright Box[C]}
\inference[]{}{}
\inference[T-BOX]{ocap(C) & Q fresh & \Gamma; x : Q \vartriangleright Box[C]; Perm[Q]; a \vdash t : \sigma}{\Gamma; a \vdash box[C] \{x \Rightarrow t \} : \bot}
\inference[]{}{}
\inference[T-CAPTURE]{\Gamma; a \vdash x : Q \vartriangleright Box[C] & \Gamma; a \vdash y : Q' \vartriangleright Box[D] \\
  \{Perm[Q], Perm[Q']\} \subseteq \Gamma & D <: ftype(C,f) \\
\Gamma \ \{Perm[Q']\}, z : Q \vartriangleright Box[C]; a \vdash t : \sigma}{\Gamma; a \vdash capture(x.f,y) \{z \Rightarrow t\} : \bot}
\inference[]{}{}
\inference[T-SWAP]{\Gamma; a \vdash x : Q \vartriangleright Box[C] & \Gamma; a \vdash y : Q' \vartriangleright Box[D'] \\
  \{Perm[Q], Perm[Q']\} \subseteq \Gamma & ftype(C,f) = Box[D] \\
  D' <: D & R fresh \\
\Gamma \ \{Perm[Q']\}, z : R \vartriangleright Box[D], Perm[R]; a \vdash t : \sigma}{\Gamma; a \vdash swap(x.f, y) \{z \Rightarrow t\} : \bot}
\inference[]{}{}
\inference[T-EMPFS]{}{H \vdash \epsilon}
\inference[]{}{}
\inference[T-FRAME1]{\Gamma; a \vdash t : \sigma & l \neq \epsilon \Longrightarrow \sigma <: C \\%
  H \vdash \Gamma; L & H \vdash \Gamma; L; P}{H \vdash \sframe{t} : \sigma}
\inference[]{}{}
\inference[T-FRAME2]{\Gamma; x : \tau; a \vdash t : \sigma & l \neq \epsilon \Longrightarrow \sigma <: C \\%
  H \vdash \Gamma; L & H \vdash \Gamma; L; P}{H \vdash^\tau_x \sframe{t} : \sigma}
\inference[]{}{}
\inference[T-FRAME-NA]{H \vdash F^\epsilon : \sigma & H \vdash FS}{H \vdash F^\epsilon \circ FS}
\inference[]{}{}
\inference[T-FRAME-NA2]{H \vdash^\tau_x F^\epsilon : \sigma & H \vdash FS}{H \vdash^\tau_x F^\epsilon \circ FS}
\inference[]{}{}
\inference[T-FRAME-A]{H \vdash F^x : \sigma & H \vdash^\sigma_x FS}{H \vdash F^x \circ FS}
\inference[]{}{}
\inference[T-FRAME-A2]{H \vdash^\tau_x F^y : \sigma & H \vdash^\sigma_y FS}{H \vdash^\tau_x F^y \circ FS}
\inference[]{}{}



\subsubsection{Evaluation}

\subsubsection{Other}
\inference[]{}{}
\inference[F-OK]{boxSep(H,F) & boxObjSep(H,F) & boxOcap(H,F) \\
a = ocap \Longrightarrow globalOcapSep(H,F) & fieldUniqueness(H,F)}{H; a \vdash F ok}

% TODO
% auxiliary predicates (boxRoot etc) appendix
% OCAP appendix
% Evaluation appenix
%




\end{document}
