\section{Inference Rules}

\NewDocumentCommand{\graybox}{ s m }{\colorbox{lightgray}{\IfBooleanT{#1}{$}#2\IfBooleanT{#1}{$}}}

% We have two kind of frames: The normal stack frames (sframe) and finish frames (fframe). They take a number of arguments:
% s:aligned? OL:variable assignments m:term OP:Permissions Ol:label
\NewDocumentCommand{\sframe}{ s O{L} m O{P} O{l} }{\langle #2,\ \IfBooleanT{#1}{&&} #3,\ \IfBooleanT{#1}{&&} #4 \IfBooleanT{#1}{&&}\rangle^#5 \IfBooleanT{#1}{&&} }
%s:aligned? m:id
\NewDocumentCommand{\fframe}{ s m }{\IfBooleanT{#1}{&&} \langle \mathit{FINISH} #2 \rangle}

% We have three kinds of reductions: frame, stack and tasks. Each come with sensible defaults
\NewDocumentCommand{\reducframe}{O{H} m O{H} m}{%
  \begin{alignedat}{7}
    #1, &#2\\\rightarrow\ #3, &#4
  \end{alignedat}%
}
\NewDocumentCommand{\reducstack}{O{H} m O{FS} s O{H} m O{FS}}{%
  \begin{alignedat}{7}
    #1, &#2 \circ #3\\\twoheadrightarrow\ #5, &#6 \circ #7
  \end{alignedat}%
}
\NewDocumentCommand{\reductasks}{m O{TS} m O{TS}}{%
  \begin{alignedat}{7}
    H, &\{#1\} && \uplus #2\\\leadsto\ H, &\{#3\} && \uplus #4
  \end{alignedat}%
}
\NewDocumentCommand{\reducasync}{m m}{%
  \begin{alignedat}{7}
    H, & #1 \\\leadsto\ H, & #2
  \end{alignedat}%
}

\subsection{Extension}
\subsubsection{Typing}
\graybox*{\inference[T-ASYNC]{
  \mathit{Perm}[Q] \in \Gamma & \Gamma \setminus \mathit{Perm}[Q]; a \vdash s : \sigma\\
  \Gamma; a \vdash b : Q \vartriangleright Box[C] & x : C; ocap \vdash t : \tau\\
}{\Gamma; a \vdash \mathit{async}(b)\{ x \Rightarrow t \} \{ s \} : \bot}}

\graybox*{\inference[T-FINISH]{
    \Gamma; ocap \vdash t : \tau
  }{\Gamma; a \vdash \mathit{finish} \{ t \} : null}}


\subsubsection{Evaluation}
\graybox*{\inference[E-ASYNC]{%
  L(b) = b(o,p) & p \in P \\
  T_1 = (f, \sframe{s}[P\setminus \{p\}][\epsilon]) &
  T_2 = (f, \sframe[[x \rightarrow o]]{t}[\emptyset][\epsilon])
}{%
  \reducasync{
    \{(f,%
      \sframe{\mathit{async}(b) \{ x \Rightarrow t \}\{s\}} \circ FS
    )\} \uplus TS%
    }{
    \{ T_1, T_2 \} \uplus TS%
    }
}}

\graybox*{\inference[E-FINISH1]{%
  T = (f', \sframe{t}[P][\epsilon]) &
  f' \mathit{fresh}
}{%
  \reductasks{%
    (f,%
      \sframe{ \mathit{let}\ x = \mathit{finish} \{\ t\ \}\ \mathit{in}\ s} \circ FS%
    )
  }{%
    (f,%
      \fframe{f'} \circ \sframe[L[x\rightarrow \mathit{null}]]{s} \circ FS%
    )
  }[\{T\} \uplus TS]
}}

\graybox*{\inference[E-FINISH2]{%
  \nexists (f', FS) \in TS
}{%
  \reductasks{%
    (f, \fframe{f'} \circ FS )%
  }{%
    (f, FS )%
  }
}}


\graybox*{\inference[E-TASK-DONE]{}{
  H,\{ (f, \epsilon) \} \uplus TS \leadsto TS
}}


\subsection{LaCasa}
\subsubsection{Well-Formedness}

\inference[WF-VAR]{%
  L(x) = \mathit{null} \vee \\%
  L(x) = o \wedge \mathit{typeof}(H,o) <: \Gamma(x) \vee \\%
  L(x) = b(o,p) \wedge \Gamma(x) = Q \vartriangleright Box[C] \wedge \mathit{typeof}(H,o) <: C%
}{H \vdash \Gamma ; L; x}

\inference[]{}{}
\inference[WF-PERM]{
  \gamma : permTypes(\Gamma) \longrightarrow P injective \\
  \forall x \in dom(\Gamma). \\
  (\Gamma(x) = Q \vartriangleright Box[C] \wedge L(x) = b(o,p) \wedge Perm[Q] \in \Gamma) \\
  \Longrightarrow \gamma(Q) = p}{\vdash \Gamma; L; P}
\inference[]{}{}
\inference[WF-ENV]{dom(\Gamma) \subseteq dom(L) \\
\forall x \in dom(\Gamma). H \vdash \Gamma; L; x}{H \vdash \Gamma; L}
\inference[]{}{}
\inference[WF-METHOD1]{\Gamma_0, this : C, x : D; \epsilon \vdash t : E' & E' <: E}{C \vdash def m(x : D) : E = t}
\inference[]{}{}
\inference[WF-METHOD2]{\Gamma = \Gamma_0, this : C, x : Q \vartriangleright Box[D], Perm[Q] & Q fresh & \Gamma; \epsilon \vdash t : E' & E' <: E}{C \vdash def m(x : Box[D]) : E = t}
\inference[]{}{}
\inference[WF-PROGRAM]{p \vdash \bar{cd} & p \vdash \Gamma_0 & \Gamma_0; \epsilon \vdash t : \sigma}{p \vdash \bar{cd} \bar{vd} t}
\inference[]{}{}
\inference[WF-CLASS]{C \vdash \bar{md} & D = AnyRef \vee p \vdash class D ... \\
  \forall (def m ...) \in \bar{md}. override(m, C, D) \\
\forall var f : \sigma \in \bar{fd}. f \notin fields(D)}{p \vdash class C extends D \{\bar{fd} \bar{md}\}}
\inference[]{}{}
\inference[WF-OVERRIDE]{mtype(m, D) not defined \vee mtype(m, D) = mtype(m, C)}{override(m, C, D)}


\subsubsection{Typing}
\inference[T-NULL]{}{\Gamma; a \vdash null : Null}
\inference[]{}{}
\inference[T-VAR]{x \in dom(\Gamma)}{\Gamma; a \vdash x : \Gamma(x)}
\inference[]{}{}
\inference[T-LET]{\Gamma; a \vdash e : \tau & \Gamma, x : \tau; a \vdash t : \sigma}{\Gamma; a \vdash let x = e in t : \sigma}
\inference[]{}{}
\inference[T-SELECT]{\Gamma; a \vdash x : C & ftype(C, f) = D}{\Gamma; a \vdash x.f : D}
\inference[]{}{}
\inference[T-ASSIGN]{\Gamma; a \vdash x : C & ftype(C, f) = D \\ \Gamma; a \vdash y : D' & D' <: D}{\Gamma; a \vdash x.f = y : D}
\inference[]{}{}
\inference[T-INVOKE]{\Gamma; a \vdash x : C & mtype(C,m) = \sigma \rightarrow \tau \\
  \Gamma; a \vdash y : \sigma' & \sigma' <: \sigma \vee \\
(\sigma = Box[D] \wedge \sigma' = Q \vartriangleright Box[D] \wedge Perm[Q] \in \Gamma)}{\Gamma; a \vdash x.m(y) : \tau}
\inference[]{}{}
\inference[T-NEW]{a = ocap \Longrightarrow ocap(C) & \forall var f : \sigma \in \bar{fd} . \exists D. \sigma = D}{\Gamma; a \vdash new C : C}
\inference[]{}{}
\inference[T-OPEN]{\Gamma; a \vdash x : Q \vartriangleright Box[C] & Perm[Q] \in \Gamma & y : C; ocap \vdash t : \sigma}{\Gamma; a \vdash x.open \{y \Rightarrow t\} : Q \vartriangleright Box[C]}
\inference[]{}{}
\inference[T-BOX]{ocap(C) & Q fresh & \Gamma; x : Q \vartriangleright Box[C]; Perm[Q]; a \vdash t : \sigma}{\Gamma; a \vdash box[C] \{x \Rightarrow t \} : \bot}
\inference[]{}{}
\inference[T-CAPTURE]{\Gamma; a \vdash x : Q \vartriangleright Box[C] & \Gamma; a \vdash y : Q' \vartriangleright Box[D] \\
  \{Perm[Q], Perm[Q']\} \subseteq \Gamma & D <: ftype(C,f) \\
\Gamma \ \{Perm[Q']\}, z : Q \vartriangleright Box[C]; a \vdash t : \sigma}{\Gamma; a \vdash capture(x.f,y) \{z \Rightarrow t\} : \bot}
\inference[]{}{}
\inference[T-SWAP]{\Gamma; a \vdash x : Q \vartriangleright Box[C] & \Gamma; a \vdash y : Q' \vartriangleright Box[D'] \\
  \{Perm[Q], Perm[Q']\} \subseteq \Gamma & ftype(C,f) = Box[D] \\
  D' <: D & R fresh \\
\Gamma \ \{Perm[Q']\}, z : R \vartriangleright Box[D], Perm[R]; a \vdash t : \sigma}{\Gamma; a \vdash swap(x.f, y) \{z \Rightarrow t\} : \bot}
\inference[]{}{}
\inference[T-EMPFS]{}{H \vdash \epsilon}
\inference[]{}{}
\inference[T-FRAME1]{\Gamma; a \vdash t : \sigma & l \neq \epsilon \Longrightarrow \sigma <: C \\%
  H \vdash \Gamma; L & H \vdash \Gamma; L; P}{H \vdash \sframe{t} : \sigma}
\inference[]{}{}
\inference[T-FRAME2]{\Gamma; x : \tau; a \vdash t : \sigma & l \neq \epsilon \Longrightarrow \sigma <: C \\%
  H \vdash \Gamma; L & H \vdash \Gamma; L; P}{H \vdash^\tau_x \sframe{t} : \sigma}
\inference[]{}{}
\inference[T-FRAME-NA]{H \vdash F^\epsilon : \sigma & H \vdash FS}{H \vdash F^\epsilon \circ FS}
\inference[]{}{}
\inference[T-FRAME-NA2]{H \vdash^\tau_x F^\epsilon : \sigma & H \vdash FS}{H \vdash^\tau_x F^\epsilon \circ FS}
\inference[]{}{}
\inference[T-FRAME-A]{H \vdash F^x : \sigma & H \vdash^\sigma_x FS}{H \vdash F^x \circ FS}
\inference[]{}{}
\inference[T-FRAME-A2]{H \vdash^\tau_x F^y : \sigma & H \vdash^\sigma_y FS}{H \vdash^\tau_x F^y \circ FS}
\inference[]{}{}



\subsubsection{Evaluation}
\inference[E-NULL]{}{\reducframe{\sframe{let x = null in t}}{\sframe[L[x \rightarrow null]]{t}}}
\inference[]{}{}
\inference[E-VAR]{}{\reducframe{\sframe{let x = y in t}}{\sframe[L[x \rightarrow L(y)]]{t}}}
\inference[]{}{}
\inference[E-SELECT]{H(L(y)) = \langle C, FM \rangle & f \in dom(FM)}{\reducframe{\sframe{let x = y.f in t}}{\sframe[L[x \rightarrow FM(f)]]{t}}}
\inference[]{}{}
\inference[E-ASSIGN]{L(y) = o & H(o) = \langle C, FM \rangle \\
  H' = H[o \rightarrow \langle C, FM[f \rightarrow L(z)]]}{
  \reducframe{\sframe{let x = y.f = z in t}}[H']{\sframe{let x = z in t}}}
\inference[]{}{}
\inference[E-NEW]{o \notin dom(H) & fields(C) = \bar{f}\\
H' = H[o \rightarrow \langle C, f \rightarrow null \rangle]}{\reducframe{\sframe{let x = new C in t}}[H']{\sframe[L[x \rightarrow o]]{t}}}
\inference[]{}{}
\inference[E-INVOKE]{H(L(y)) = \langle C, FM \rangle & mbody(C, m) = x \rightarrow t'\\
  L' = L_0[this \rightarrow L(y), x \rightarrow L(z)]\\
P' = \emptyset \vee (L(z) = b(o, p) \wedge p \in P \wedge P' = \{ p \})}{\reducstack{\sframe{let x = y.m(z) in t}}{\sframe[L']{t'}[P'][x] \circ \sframe{t}}}
\inference[]{}{}
\inference[E-RETURN1]{}{\reducframe{\sframe{x}[P][y] \circ \sframe[L']{t'}[P']}{\sframe[L'[y \rightarrow L(x)]]{t'}[P']}}
\inference[]{}{}
\inference[E-RETURN2]{}{\reducframe{\sframe{x}[P][\epsilon] \circ \sframe[L']{t'}[P']}{\sframe[L']{t'}[P']}}
\inference[]{}{}
\inference[E-OPEN]{L(y) = b(o,p) & p \in P & L'= [z \rightarrow o]}{\reducstack{
    \sframe{let x = y.open \{ z \Rightarrow t'\} in t}
  }{
    \sframe[L']{t'}[\emptyset][\epsilon] \circ \sframe[L[x \rightarrow L(y)]]{t}
  }}
\inference[]{}{}
\inference[E-BOX]{o \notin dom(H) & fields(C) = \bar{f}\\
  H' = H[ o \rightarrow \langle C, f \rightarrow null \rangle ] & p fresh}{
  \reducstack{
    \sframe{box[C]\{x \Rightarrow t\}}
  }*[H']{
    \sframe[L[x \rightarrow b(o,p)]]{t}[P \cup \{p\}][\epsilon]
  }[\epsilon]
}
\inference[]{}{}
\inference[E-CAPTURE]{L(x) = b(o,p) & L(y) = b(o', p') & \{p, p'\} \subseteq P\\
  H(o) = \langle C, FM \rangle & H' = H[ \rightarrow \langle C, FM[ f \rightarrow o' ] \rangle ]}{
  \reducstack{
    \sframe{capture(x.f, y) \{ z \Rightarrow t\}}
  }*[H']{
    \sframe[L[z \rightarrow L(x)]]{t}[P \setminus \{p'\}][\epsilon]
  }[\epsilon]
}
\inference[]{}{}
\inference[E-SWAP]{L(x) = b(o,p) & L(y) = b(o', p') & \{p, p'\} \subseteq P\\
  H(o) = \langle C, FM \rangle & FM(f) = o'' & p'' fresh\\
  H' = H[o \rightarrow \langle C, FM[f \rightarrow o'] \rangle ]}{
  \reducstack{
    \sframe{swap(x.f,y) \{ z \Rightarrow t\}}
  }*[H']{
    \sframe[L[z \rightarrow b(o'', p'')]]{t}[(P \setminus \{ p' \}) \cup \{p''\}][\epsilon]
  }[\epsilon]
}
\inference[]{}{}

\subsubsection{Definitions}
\begin{definition}[Object Type]
  For an object identifier $o \in dom(H)$ where $H(o) = \langle C, FM \rangle$, $typeof(H, o) := C$
\end{definition}

\begin{definition}[Well-typed Heap]
A heap $H$ is well-typed, written $\vdash H : \star$, iff
\begin{equation}
\begin{split}
  \forall o \in dom(H). & H(o) = \langle C, FM \rangle \Longrightarrow\\
                        & (dom(FM) = fields(C) \wedge\\
                        & \forall f \in dom(FM). FM(f)= null \vee typeof(H, FM(f)) <: ftype(C, f))
\end{split}
\end{equation}
\end{definition}

\begin{definition}[Separation]
  Two object identifiers $o$ and $o'$ are separate in heap $H$, written $sep(H, o, o')$, iff $\forall q, q' \in dom(H). reach(H, o, q) \wedge reach(H, o', q') \Longrightarrow q \neq q'$.
\end{definition}

\subsubsection{Other}
\inference[ACC-F]{x \rightarrow o \in L \vee (x \rightarrow b(o,p) \in L \wedge p \in P)}{accRoot(o, \langle L, t, P \rangle ^ l)}
\inference[]{}{}
\inference[ACC-FS]{accRoot(o, F) \vee accRoot(o, FS)}{accRoot(o, F \circ FS)}
\inference[]{}{}
\inference[ISO-FS]{
  \forall o, o' \in dom(H). (accRoot(o, FS) \wedge accRoot(o', FS')) \Rightarrow sep(H, o, o')\\
}{isolated(H, FS, FS')}
\inference[]{}{}
\graybox*{\inference[ISO-TS]{
  \forall (f, FS), (f', FS') \in TS. FS \neq FS' \Rightarrow isolated(H, FS, FS') \vee\\
  FS = \fframe{f'} \circ GS \vee FS' = \fframe{f} \circ GS'
}{isolated(H, TS)}}

\inference[]{}{}
\inference[F-OK]{boxSep(H,F) & boxObjSep(H,F) & boxOcap(H,F) \\
a = ocap \Longrightarrow globalOcapSep(H,F) & fieldUniqueness(H,F)}{H; a \vdash F ok}
\inference[]{}{}
\inference[SINGFS-OK]{H; a \vdash F ok}{H; a \vdash F \circ \epsilon ok}
\inference[]{}{}
\inference[FS-OK]{
  H; b \vdash F^l ok & H; a \vdash FS ok
}{H; b \vdash F^l \circ FS ok}
\subsubsection{Predicates}
\inference[]{x \rightarrow b(o,p) \in L & p \in P}{boxRoot(o, \sframe{t})}
\inference[]{}{}
\inference[]{boxRoot(o, F)}{boxRoot(o, F \circ \epsilon)}
\inference[]{}{}
\inference[]{boxRoot(o, F) \vee boxRoot(o, FS)}{boxRoot(o, F \circ FS)}
\inference[]{}{}
\inference[]{}{}
\inference[]{x \rightarrow b(o,p) \in L & p \in P}{boxRoot(o, \sframe{t}, p)}
\inference[]{}{}
\inference[]{boxRoot(o, F, p)}{boxRoot(o, F \circ \epsilon, p)}
\inference[]{}{}
\inference[]{boxRoot(o, F, p) \vee boxRoot(o, FS, p)}{boxRoot(o, F \circ FS, p)}
\inference[]{}{}
\inference[]{boxRoot(o, FS) & x \rightarrow o' \in env(F) & reach(H, o, o')}{openbox(H, o, F, FS)}
\inference[]{}{}

% TODO
% OCAP appendix
%
