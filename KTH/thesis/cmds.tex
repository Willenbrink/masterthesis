% Variable name vn, monospaced name mn, boldfaced name bn
\NewDocumentCommand{\vn}{m}{\mathop{\mathit{#1}}}
\NewDocumentCommand{\mn}{m}{\texttt{#1}}
\NewDocumentCommand{\bn}{m}{\mathop{\textbf{#1}}}

% TODO spaces afterwards don't really work
\NewDocumentCommand{\plc}{}{\textbf{PLC}\xspace}

\NewDocumentCommand{\FS}{}{\vn{FS}}
\NewDocumentCommand{\TS}{}{\vn{TS}}
\NewDocumentCommand{\WS}{}{\vn{WS}}

\newcommand{\reducedstrut}{\vrule width 0pt height .9\ht\strutbox depth .9\dp\strutbox\relax}
\NewDocumentCommand
    {\graybox}
    { s m s }
    {\begingroup%
    \IfBooleanT{#3}{\setlength{\fboxsep}{1pt}}%
    \colorbox%
        {lightgray}%
        {\reducedstrut\IfBooleanT{#1}{$}#2\IfBooleanT{#1}{$}\/}%
    \endgroup%
    }

% We have two kind of frames: The normal stack frames (sframe) and finish frames (fframe). They take a number of arguments:
% s:aligned? OL:variable assignments m:term OP:Permissions Ol:label
\NewDocumentCommand{\sframe}{ s O{L} m O{P} O{l} }{\langle #2,\ \IfBooleanT{#1}{&&} #3,\ \IfBooleanT{#1}{&&} #4 \IfBooleanT{#1}{&&}\rangle^#5 \IfBooleanT{#1}{&&} }
%s:aligned? m:id
%\NewDocumentCommand{\fframe}{ s m O{l} }{\IfBooleanT{#1}{&&} \langle \mathit{FINISH} #2 \rangle ^#3 }

% We have three kinds of reductions: frame, stack and tasks. Each come with sensible defaults
\NewDocumentCommand{\reducframe}{O{H} m s O{H} m s}{%
  \begin{alignedat}{7}
    &#1, #2\IfBooleanF{#6}{\\} \rightarrow\ &#4, #5
  \end{alignedat}%
}
\NewDocumentCommand{\reducstack}{O{H} m O{FS} s O{H} m O{FS} s}{%
  \begin{alignedat}{7}
    &#1, \IfBlankF{#2}{#2 \circ} #3\IfBooleanF{#8}{\\} \twoheadrightarrow\ &#5, \IfBlankF{#6}{#6 \circ} #7
  \end{alignedat}%
}
% task, taskset+waitset, s, heap, new task, new taskset+waitset, linebreak
\NewDocumentCommand{\reductasks}{m O{\TS, \WS} s O{H} m O{\TS, \WS} s}{%
  \begin{alignedat}{7}
    &H, \IfBlankF{#1}{\{#1\} \uplus} #2 \IfBooleanF{#7}{\\} \leadsto\ &#4, \IfBlankF{#5}{\{#5\} \uplus} #6
  \end{alignedat}
}

\NewDocumentCommand{\IfReport}{m}{#1}
%\NewDocumentCommand{\IfReport}{m}{}

\newtheorem{theorem}{Theorem}
\newtheorem{lemma}[theorem]{Lemma}
\theoremstyle{definition}
\newtheorem{definition}{Definition}[section]

\newtheorem*{futuretheoreminner}{Theorem~\thefuturetheoreminner}
\newcommand{\thefuturetheoreminner}{} % initialize

\ExplSyntaxOn

\prop_new:N \g_alevel_future_prop

\NewDocumentEnvironment{futuretheorem}{ m o +b }
 {% #1 is the label to be also used in the body for restating
  % #2 is the standard optional argument for a theorem
  % #3 is the body
  \renewcommand{\thefuturetheoreminner}{\ref{#1}}
  \IfNoValueTF{#2}
   {\futuretheoreminner}
   {\futuretheoreminner[#2]}
  #3
 }
 {
  \endfuturetheoreminner
  \prop_gput:Nnn \g_alevel_future_prop { #1 } { #3 }
  \IfValueT{#2}{ \prop_gput:Nnn \g_alevel_future_prop { #1-attr } { #2 } }
 }

\NewDocumentCommand{\pasttheorem}{m}
 {
  \prop_if_in:NnTF \g_alevel_future_prop { #1-attr }
   {
    \begin{theorem}[\prop_item:Nn \g_alevel_future_prop { #1-attr }]
   }
   {
    \begin{theorem}
   }
  \label{#1}
  \prop_item:Nn \g_alevel_future_prop { #1 }
  \end{theorem}
 }

\ExplSyntaxOff